% \iffalse
%<*driver>
\documentclass[full]{l3doc}
\usepackage{hologo}
\begin{filecontents}[overwrite,nosearch,noheader]{\jobname.bib}

@online{expl3,
  title = {expl3},
  subtitle = {Wrapper package for experimental \LaTeX3},
  author = {{The \LaTeX{} Team}},
  publisher = {{CTAN}},
  date = {2022-06-16},
  urldate = {2022-06-26},
  url = {https://ctan.org/pkg/expl3},
}

@online{markdown,
  title = {Markdown},
  subtitle = {A package for converting and rendering markdown documents inside \TeX},
  author = {Vít Novotný},
  publisher = {{CTAN}},
  date = {2022-05-31},
  urldate = {2022-06-26},
  url = {https://ctan.org/pkg/markdown},
  note = {Version 2.15.2-0-gb238dbc},
}

\end{filecontents}
\usepackage{biblatex}
\addbibresource{\jobname.bib}
\begin{document}
  \DocInput{\jobname.dtx}
\end{document}
%</driver>
% \fi
%
% \title{^^A
%   The \textsf{lt3luabridge} package: \Lua{} without \Lua\TeX^^A
% }
%
% \author{^^A
%  Vít Novotný\thanks
%    {^^A
%      E-mail:
%        \href{mailto:witiko@mail.muni.cz}
%          {witiko@mail.muni.cz}^^A
%    }^^A
% }
%
% \date{Released 2022-06-26}
%
% \maketitle
%
% \begin{documentation}
%
% The \pkg{lt3luabridge} expl3~\cite{expl3} package provides support for
% executing \Lua{} code in \Lua\TeX{} or any other \TeX{} engine that exposes
% the shell. The package provides interfaces to plain \TeX, \LaTeX, and
% \Hologo{ConTeXt}
% formats:
% \begin{verbatim}
% \documentclass{standalone}
% \usepackage{lt3luabridge}
% \begin{document}
% $ 1 + 2 = \luabridgeExecute{ print(1 + 2) } $
% \end{document}
% \end{verbatim}
% The package was previously part of the Markdown package~\cite{markdown},
% where it has been battle-tested since 2016.  Since 2022, lt3luabridge has
% also been available as a separate package.
%
% \section{Loading the package}
%
% Use the |\input lt3luabridge\relax| command to load the package from plain \TeX,
% use the |\usepackage{lt3luabridge}| command to load the package from \LaTeX, and
% use the |\usemodule[t][lt3luabridge]| command to load the package from \Hologo{ConTeXt}.
%
% \section{Executing \Lua{} code}
%
% The interface for executing \Lua{} code mimics the \cs{lua_now:n} function
% from \pkg{l3luatex}.
%
% \begin{function}[added = 2022-06-26]{\luabridge_now:n, \luabridge_now:e}
%   \begin{syntax}
%     \cs{luabridge_now:n} \Arg{token list}
%   \end{syntax}
%   The \meta{token list} is first tokenized by \TeX{}, which includes
%   converting line ends to spaces in the usual \TeX{} manner and which
%   respects currently-applicable \TeX{} category codes. The resulting
%   \meta{\Lua{} input} is passed to the \Lua{} interpreter for processing.
%   Each \cs{luabridge_now:n} block is treated by \Lua{} as a separate chunk.
%   The \Lua{} interpreter executes the \meta{\Lua{} input} immediately,
%   and in an expandable manner.
%
%   Unlike \cs{lua_now:n}, \cs{luabridge_now:n} may execute \meta{\Lua{} input}
%   in a separate process from \TeX. Therefore, you should not interact with
%   \TeX{} from \meta{\Lua{} input}. The only exception is the standard output
%   produced by \meta{\Lua{} input} using for example the |print()| \Lua{}
%   function like in the example at the top of this page. The standard output
%   will be inserted into \TeX's input stream after \meta{\Lua{} input} has
%   been processed at the latest.
% \end{function}
%
% \begin{function}[added = 2022-06-26]{\luabridgeExecute}
%   \begin{syntax}
%     \cs{luabridgeExecute} \Arg{token list}
%   \end{syntax}
%   The \cs{luabridgeExecute} document command aliases
%   the \cs{luabridge_now:e} function.
% \end{function}
%
% \section{Setting and getting the method to execute \Lua{} code}
%
% There are several methods that can be used to execute \Lua{} code. This
% section describes the interface that the package provides to set the
% preferred method or to determine which method was used.
%
% \begin{variable}[added = 2022-06-26]{\g_luabridge_method_int}
%   This variable controls the method used to execute \Lua{} code.
%   The variable is set automatically when the package is loaded
%   and changing the value of the variable afterwards has no effect.
%   However, we can set the value of the variable before loading the
%   package to one of the constants described below.
% \end{variable}
%
% \begin{variable}[added = 2022-06-26]{\c_luabridge_method_write_eighteen_int}
%   Use shell escape through the \cs{write18} \TeX{} command to execute \Lua{} code.
% \end{variable}
%
% \begin{variable}[added = 2022-06-26]{\c_luabridge_method_os_execute_int}
%   Use shell escape through the |os.execute()| Lua function to execute \Lua{} code.
% \end{variable}
%
% \begin{variable}[added = 2022-06-26]{\c_luabridge_method_directlua_int}
%   Use the \cs{directlua} primitive of Lua\TeX{} to execute \Lua{} code.
% \end{variable}
%
% \section{Setting and getting the filenames of helper files}
%
% When shell escape is used to execute \Lua{} code, several helper files are
% needed to shuffle around code and output. The following variables and
% constants are undefined when the \cs{directlua} primitive of Lua\TeX{} is
% used to execute \Lua{} code.
%
% \begin{variable}[added = 2022-06-26]{\g_luabridge_output_dirname_str}
%   This variable controls the output directory that will store the helper
%   files. The variable should be set to the same value as the
%   \texttt{-output-directory} parameter of the \TeX{} engine.
% \end{variable}
%
% \begin{variable}[added = 2022-06-26]{\c_luabridge_default_output_dirname_str}
%   This constant is the default value of \cs{g_luabridge_output_dirname_str}.
% \end{variable}
%
% \begin{variable}[added = 2022-06-26]{\g_luabridge_helper_script_filename_str}
%   This variable controls the filename of a helper \Lua{} script that will be
%   executed from the shell using the \TeX{} \Lua{} interpreter.
% \end{variable}
%
% \begin{variable}[added = 2022-06-26]{\c_luabridge_default_helper_script_filename_str}
%   This constant is the default value of
%   \cs{g_luabridge_helper_script_filename_str}.
% \end{variable}
%
% \begin{variable}[added = 2022-06-26]{\g_luabridge_standard_output_filename_str}
%   This variable controls the filename of a helper file that will contain the
%   standard output produced by the |texlua| interpreter (if any).
% \end{variable}
%
% \begin{variable}[added = 2022-06-26]{\c_luabridge_default_standard_output_filename_str}
%   This constant is the default value of
%   \cs{g_luabridge_standard_output_filename_str}.
% \end{variable}
%
% \begin{variable}[added = 2022-06-26]{\g_luabridge_error_output_filename_str}
%   This variable controls the filename of a helper file that will contain the
%   error output produced by the |texlua| interpreter (if any).
% \end{variable}
%
% \begin{variable}[added = 2022-06-26]{\c_luabridge_default_error_output_filename_str}
%   This constant is the default value of
%   \cs{g_luabridge_error_output_filename_str}.
% \end{variable}
%
% \end{documentation}
%
% \begin{implementation}
%
% \section{Plain \TeX{} implementation}
%
% This section contains the implementation for plain \TeX{} using generic expl3.
%
%    \begin{macrocode}
%<@@=luabridge>
%<*generic-package>
\ifx\ExplSyntaxOn\undefined
  \input expl3-generic\relax
\fi
\ExplSyntaxOn
\int_const:Nn
  \c_luabridge_method_write_eighteen_int
  { 0 }
\int_const:Nn
  \c_luabridge_method_os_execute_int
  { 1 }
\int_const:Nn
  \c_luabridge_method_directlua_int
  { 2 }
\int_if_exist:NF
  \g_luabridge_method_int
  {
    \int_new:N
      \g_luabridge_method_int
    \sys_if_engine_luatex:TF
      {
        \int_gset_eq:NN
          \g_luabridge_method_int
          \c_luabridge_method_directlua_int
      }
      {
        \int_gset_eq:NN
          \g_luabridge_method_int
          \c_luabridge_method_write_eighteen_int
      }
  }
\msg_new:nnn
  { luabridge }
  { unknown-method }
  {
    Unknown~bridging~method:~#1
  }
\msg_new:nnn
  { luabridge }
  { method-write-eighteen }
  {
    Using~shell~escape~via~write18~as~the~bridging~method
  }
\msg_new:nnn
  { luabridge }
  { method-os-execute }
  {
    Using~shell~escape~via~os.execute()~as~the~bridging~method
  }
\msg_new:nnn
  { luabridge }
  { method-directlua }
  {
    Using~direct~Lua~access~as~the~bridging~method
  }
\int_case:nnF
  { \g_luabridge_method_int }
  {
    { \c_luabridge_method_write_eighteen_int }
      {
        \msg_info:nn
          { luabridge }
          { method-write-eighteen }
      }
    { \c_luabridge_method_os_execute_int }
      {
        \msg_info:nn
          { luabridge }
          { method-os-execute }
      }
    { \c_luabridge_method_directlua_int }
      {
        \msg_info:nn
          { luabridge }
          { method-directlua }
      }
  }
  {
    \cs_generate_variant:Nn
      \msg_error:nnn
      { nnV }
    \msg_error:nnV
      { luabridge }
      { unknown-method }
      \g_luabridge_method_int
  }
\bool_if:nTF
  {
    \int_compare_p:nNn
      { \g_luabridge_method_int }
      =
      { \c_luabridge_method_write_eighteen_int } ||
    \int_compare_p:nNn
      { \g_luabridge_method_int }
      =
      { \c_luabridge_method_os_execute_int }
  }
  {
    \int_const:Nn
      \c_@@_level_disabled_int
      { 0 }
    \int_const:Nn
      \c_@@_level_enabled_int
      { 1 }
    \int_const:Nn
      \c_@@_level_restricted_int
      { 2 }
    \int_new:N
      \l_@@_level_int
    \cs_if_exist:NTF
      \pdfshellescape
      {
        \int_gset:Nn
          \l_@@_level_int
          { \pdfshellescape }
      }
      {
        \cs_if_exist:NTF
          \shellescape
          {
            \int_gset:Nn
              \l_@@_level_int
              { \shellescape }
          }
          {
            \int_case:nnF
              { \g_luabridge_method_int }
              {
                { \c_luabridge_method_write_eighteen_int }
                  {
                    \int_gset_eq:NN
                      \l_@@_level_int
                      \c_@@_level_enabled_int
                  }
              }
              {
                \int_gset:Nn
                  \l_@@_level_int
                  {
                    \lua_now:n
                      {
                        tex.sprint(status.shell_escape or "1")
                      }
                  }
              }
          }
      }
    \msg_new:nnn
      { luabridge }
      { unknown-level }
      {
        Unknown~shell~escape~level:~#1
      }
    \msg_new:nnnn
      { luabridge }
      { level-disabled }
      {
        Shell~escape~seems~to~be~disabled
      }
      {
        You~may~need~to~run~TeX~with~the~--shell-escape~or~the~
        --enable-write18~flag,~or~write~shell_escape=t~in~the~
        texmf.cnf~file.
      }
    \msg_new:nnn
      { luabridge }
      { level-enabled }
      {
        Shell~escape~seems~to~be~enabled
      }
    \msg_new:nnnn
      { luabridge }
      { level-restricted }
      {
        Shell~escape~seems~to~be~restricted
      }
      {
        You~may~need~to~run~TeX~with~the~--shell-escape~or~the~
        --enable-write18~flag,~or~write~shell_escape=t~in~the~
        texmf.cnf~file.
      }
    \str_const:Nn
      \c_luabridge_default_output_dirname_str
      { . }
    \str_const:Nx
      \c_luabridge_default_helper_script_filename_str
      { \jobname.luabridge.lua }
    \str_const:Nx
      \c_luabridge_default_error_output_filename_str
      { \jobname.luabridge.err }
    \str_const:Nx
      \c_luabridge_default_standard_output_filename_str
      { \jobname.luabridge.out }
    \int_case:nnF
      { \l_@@_level_int }
      {
        { \c_@@_level_disabled_int }
          {
            \msg_warning:nn
              { luabridge }
              { level-disabled }
          }
        { \c_@@_level_enabled_int }
          {
            \msg_info:nn
              { luabridge }
              { level-enabled }
          }
        { \c_@@_level_restricted_int }
          {
            \msg_warning:nn
              { luabridge }
              { level-restricted }
          }
      }
      {
        \msg_error:nnx
          { luabridge }
          { unknown-level }
          { \l_@@_level_int }
      }
    \cs_new:Nn
      \_luabridge_assert_shell_escape:
      {
        \int_case:nnF
          { \l_@@_level_int }
          {
            { \c_@@_level_disabled_int }
              {
                \msg_error:nn
                  { luabridge }
                  { level-disabled }
              }
          }
      }
    \int_case:nn
      { \g_luabridge_method_int }
      {
        { \c_luabridge_method_write_eighteen_int }
          {
            \cs_new:Nn
              \_luabridge_execute_shell:n
              {
                \_luabridge_assert_shell_escape:
                \immediate
                  \write 18
                    { #1 }
              }
          }
        { \c_luabridge_method_os_execute_int }
          {
            \cs_new:Nn
              \_luabridge_execute_shell:n
              {
                \_luabridge_assert_shell_escape:
                \lua_now:e
                  {
                    os.execute(
                      " \lua_escape:e { #1 } "
                    )
                  }
              }
          }
      }
    \str_if_exist:NF
      \g_luabridge_output_dirname_str
      {
        \str_new:N
          \g_luabridge_output_dirname_str
        \tl_gset:Nn
          \g_luabridge_output_dirname_str
          \c_luabridge_default_output_dirname_str
      }
    \str_if_exist:NF
      \g_luabridge_helper_script_filename_str
      {
        \str_gset_eq:NN
          \g_luabridge_helper_script_filename_str
          \c_luabridge_default_helper_script_filename_str
      }
    \str_if_exist:NF
      \g_luabridge_error_output_filename_str
      {
        \str_gset_eq:NN
          \g_luabridge_error_output_filename_str
          \c_luabridge_default_error_output_filename_str
      }
    \str_if_exist:NF
      \g_luabridge_standard_output_filename_str
      {
        \str_gset_eq:NN
          \g_luabridge_standard_output_filename_str
          \c_luabridge_default_standard_output_filename_str
      }
    \cs:w newwrite \cs_end:
      \l_@@_output_stream
    \cs_new:Nn
      \luabridge_now:n
      {
        \immediate \openout
          \l_@@_output_stream
          \g_luabridge_helper_script_filename_str
        \msg_info:nnV
          { luabridge }
          { writing-helper-script }
          \g_luabridge_helper_script_filename_str
        \tl_set:Nn
          \l_tmpa_tl
          { #1 }
        \tl_set:Nx
          \l_tmpb_tl
          {
            local~ran_ok,~error~=~pcall(function()~
              local~ran_ok,~kpse~=~pcall(require,~"kpse")~
              if~ran_ok~then~kpse.set_program_name("luatex")~end~
              \exp_not:V \l_tmpa_tl~
            end)~
            if~not~ran_ok~then~
              local~file~=~io.open("
                \g_luabridge_output_dirname_str /
                \g_luabridge_error_output_filename_str
              ",~"w")~
              if~file~then~
                file:write(error~..~" \iow_char:N \\ n ")~
                file:close()~
              end~
              print('
                \iow_char:N \\ \iow_char:N \\ begingroup
                  \iow_char:N \\ \iow_char:N \\ ExplSyntaxOn
                  \iow_char:N \\ \iow_char:N \\ msg_error:nnvv
                    { luabridge }
                    { failed-to-execute }
                    { g_luabridge_output_dirname_str }
                    { g_luabridge_output_dirname_str }
                \iow_char:N \\ \iow_char:N \\ endgroup
              ')~
            end
          }
        \immediate \write
          \l_@@_output_stream
          { \exp_not:V \l_tmpb_tl }
        \immediate \closeout
          \l_@@_output_stream
        \msg_info:nnVV
          { luabridge }
          { executing-helper-script }
          \g_luabridge_helper_script_filename_str
          \g_luabridge_standard_output_filename_str
        \tl_set:Nx
          \l_tmpa_tl
          {
            texlua~"
              \g_luabridge_output_dirname_str /
              \g_luabridge_helper_script_filename_str
            "~>~"
              \g_luabridge_output_dirname_str /
              \g_luabridge_standard_output_filename_str
            "
          }
        \_luabridge_execute_shell:V
          \l_tmpa_tl
        \file_if_exist_input:VF
          \g_luabridge_standard_output_filename_str
          {
            \msg_error:nn
              { luabridge }
              { level-disabled }
          }
      }
    \cs_generate_variant:Nn
      \msg_info:nnn
      { nnV }
    \cs_generate_variant:Nn
      \msg_info:nnnn
      { nnVV }
    \cs_generate_variant:Nn
      \msg_error:nnnn
      { nnvv }
    \cs_generate_variant:Nn
      \_luabridge_execute_shell:n
      { V }
    \prg_generate_conditional_variant:Nnn
      \file_if_exist_input:n
      { V }
      { F }
    \msg_new:nnn
      { luabridge }
      { writing-helper-script }
      {
        Writing~a~helper~Lua~script~to~file~#1
      }
    \msg_new:nnn
      { luabridge }
      { executing-helper-script }
      {
        Executing~a~helper~Lua~script~from~file~#1~
        and~storing~the~result~in~file~#2
      }
    \msg_new:nnnn
      { luabridge }
      { failed-to-execute }
      {
        An~error~was~encountered~while~executing~Lua~code
      }
      {
        For further clues, examine file #1/#2
      }
  }
  {
    \cs_new:Nn
      \luabridge_now:n
      {
        \tl_set:Nn
          \l_tmpa_tl
          { #1 }
        \tl_set:Nx
          \l_tmpb_tl
          {
            local~function~print(input)~
              input~=~tostring(input)~
              local~output~=~{}~
              for~line~in~input:gmatch("[^
                    \iow_char:N \\ r
                    \iow_char:N \\ n
                  ]+")~do~
                table.insert(output,~line)~
              end~
              tex.print(output)~
            end~
            \exp_not:V \l_tmpa_tl
          }
        \lua_now:V
          \l_tmpb_tl
      }
    \cs_generate_variant:Nn
      \lua_now:n
      { V }
  }
\cs_new_protected:Npn
  \luabridgeExecute
  #1
  {
    \luabridge_now:e
      { #1 }
  }
\cs_generate_variant:Nn
  \luabridge_now:n
  { e }
\ExplSyntaxOff
%</generic-package>
%    \end{macrocode}
%
% \section{\LaTeX{} implementation}
%
% This section contains the implementation for \LaTeX.
%
%    \begin{macrocode}
%<*latex-package>
\RequirePackage{expl3}
\ProvidesExplPackage
  {lt3luabridge}%
  {2022-06-26}%
  {1.0.1}%
  {An expl3 package that allows you to execute Lua code in LuaTeX or any other
   TeX engine that exposes the shell}
\input lt3luabridge\relax
%</latex-package>
%    \end{macrocode}
%
% \section{\Hologo{ConTeXt} implementation}
%
% This section contains the implementation for \Hologo{ConTeXt}.
% \Hologo{ConTeXt} MkII, MkIV, and later formats are supported.
%
%    \begin{macrocode}
%<*context-package>
\writestatus{loading}{ConTeXt User Module / lt3luabridge}
\startmodule[lt3luabridge]
\unprotect
\input lt3luabridge\relax
%</context-package>
%    \end{macrocode}
%
% \end{implementation}
%
% \printbibliography
% \PrintIndex
